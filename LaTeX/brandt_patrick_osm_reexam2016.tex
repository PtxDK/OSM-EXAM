% Document class: article with font size 11pt
% ---------------
\documentclass[11pt,a4paper]{article}

\setlength{\textwidth}{165mm}
\setlength{\textheight}{240mm}
\setlength{\parindent}{0mm} % S{\aa} meget rykkes ind efter afsnit
\setlength{\parskip}{\baselineskip}
\setlength{\headheight}{0mm}
\setlength{\headsep}{0mm}
\setlength{\hoffset}{-2.5mm}
\setlength{\voffset}{0mm}
\setlength{\footskip}{15mm}
\setlength{\oddsidemargin}{0mm}
\setlength{\topmargin}{0mm}
\setlength{\evensidemargin}{0mm}

\usepackage[a4paper, hmargin={2.8cm, 2.8cm}, vmargin={2.5cm, 2.5cm}]{geometry}
\usepackage[super]{nth}
\PassOptionsToPackage{hyphens}{url}\usepackage{hyperref}
\usepackage{eso-pic} % \AddToShipoutPicture
\usepackage{float} % This will allow precise picture placement, use [H].
\usepackage{listings}
\usepackage{color}
\usepackage[titletoc,title]{appendix}

% BEGIN REDESIGN OF LSTLISTING:
\definecolor{codegreen}{rgb}{0,0.6,0}
\definecolor{codegray}{rgb}{0.5,0.5,0.5}
\definecolor{codepurple}{rgb}{0.58,0,0.82}
\definecolor{backcolour}{rgb}{0.95,0.95,0.92}

\lstdefinestyle{mystyle}{
    backgroundcolor=\color{backcolour},
    commentstyle=\color{codegreen},
    keywordstyle=\color{magenta},
    numberstyle=\tiny\color{codegray},
    stringstyle=\color{codepurple},
    basicstyle=\footnotesize,
    breakatwhitespace=false,
    breaklines=true,
    captionpos=b,
    keepspaces=true,
    numbers=left,
    numbersep=5pt,
    showspaces=false,
    showstringspaces=false,
    showtabs=false,
    tabsize=2
}

\lstset{style=mystyle}
% END REDESIGN OF LSTLISTING


% Call packages
% ---------------
\usepackage{comment} %Possible to comment larger sections
%http://get-software.net/macros/latex/contrib/comment/comment.pdf
\usepackage[T1]{fontenc} %oriented to output, that is, what fonts to use for printing characters.
\usepackage[utf8]{inputenc} %allows the user to input accented characters directly from the keyboard

%Support Windows TeXStudio
\usepackage[T1]{fontenc}
\usepackage{lmodern}

%http://mirrors.dotsrc.org/ctan/fonts/fourier-GUT/doc/latex/fourier/fourier-doc-en.pdf
\usepackage[english]{babel}														     % Danish
\usepackage[protrusion=true,expansion=true]{microtype}				                 % Better typography
%http://www.khirevich.com/latex/microtype/
\usepackage{amsmath,amsfonts,amsthm, amssymb}							 % Math packages
\usepackage[pdftex]{graphicx} %puts to pdf and graphic
%http://www.kwasan.kyoto-u.ac.jp/solarb6/usinggraphicx.pdf
\usepackage{xcolor,colortbl}
%http://mirrors.dotsrc.org/ctan/macros/latex/contrib/xcolor/xcolor.pdf
%http://texdoc.net/texmf-dist/doc/latex/colortbl/colortbl.pdf
\usepackage{tikz} %documentation http://www.ctan.org/pkg/pgf
\usepackage{parskip} %http://www.ctan.org/pkg/parskip
%http://tex.stackexchange.com/questions/51722/how-to-properly-code-a-tex-file-or-at-least-avoid-badness-10000
%Never use \\ but instead press "enter" twice. See second website for more info

% MATH -------------------------------------------------------------------
\newcommand{\Real}{\mathbb R}
\newcommand{\Complex}{\mathbb C}
\newcommand{\Field}{\mathbb F}
\newcommand{\RPlus}{[0,\infty)}
%
\newcommand{\norm}[1]{\left\Vert#1\right\Vert}
\newcommand{\essnorm}[1]{\norm{#1}_{\text{\rm\normalshape ess}}}
\newcommand{\abs}[1]{\left\vert#1\right\vert}
\newcommand{\set}[1]{\left\{#1\right\}}
\newcommand{\seq}[1]{\left<#1\right>}
\newcommand{\eps}{\varepsilon}
\newcommand{\To}{\longrightarrow}
\newcommand{\RE}{\operatorname{Re}}
\newcommand{\IM}{\operatorname{Im}}
\newcommand{\Poly}{{\cal{P}}(E)}
\newcommand{\EssD}{{\cal{D}}}
% THEOREMS ----------------------------------------------------------------
\theoremstyle{plain}
\newtheorem{thm}{Theorem}[section]
\newtheorem{cor}[thm]{Corollary}
\newtheorem{lem}[thm]{Lemma}
\newtheorem{prop}[thm]{Proposition}
%
\theoremstyle{definition}
\newtheorem{defn}{Definition}[section]
%
\theoremstyle{remark}
\newtheorem{rem}{Remark}[section]
%
\numberwithin{equation}{section}
\renewcommand{\theequation}{\thesection.\arabic{equation}}


\author{
  \Large{
    Brandt, Patrick Krøll - bwx155} \\
   \\
   %\Large{ }
}
\title{
  \huge{OSM 2016 \\}
  \Large{Operating systems and multiprogramming \\}
  \vspace{3cm}
  \Large{OSM Re-Exam}
}

\begin{document}

\AddToShipoutPicture*{\put(0,0){\includegraphics*[viewport=0 0 700 600]{include/natbio-farve}}}
\AddToShipoutPicture*{\put(0,602){\includegraphics*[viewport=0 600 700 1600]{include/natbio-farve}}}

\AddToShipoutPicture*{\put(0,0){\includegraphics*{include/nat-en}}}

\clearpage\maketitle
\thispagestyle{empty}
\clearpage\newpage
\thispagestyle{plain}

% \tableofcontents
% \pagebreak

%<<--------------------------------------------------------------->>
\section*{Theoretical 1: Critical Intersection}

\textit{To shortly introduce the my solution, I will explain the assumptions that is made to make it a valid solution}

The XML code for onlineta.github.io can be found in t1.xml




\section*{Theoretical 2: Scheduling}

\begin{table}[H]
	\centering
	\caption{Basic Info Table}
	\label{sch-basic}
	\begin{tabular}{lll}
		\hline
		\multicolumn{1}{|l|}{PID} & \multicolumn{1}{l|}{AT} & \multicolumn{1}{l|}{BT} \\ \hline
		1                         & 0                       & 100                     \\
		2                         & 1                       & 10                      \\
		3                         & 2                       & 20                      \\
		4                         & 3                       & 1                       \\
		5                         & 10                      & 1                       \\
		6                         & 20                      & 100                    
	\end{tabular}
\end{table}


Completion time was found with: $PreviousCT+RemainingBT$

Turnaround time was found with $TAT = CT-AT$

Waiting/response time was found with $WT = TAT-BT$


\subsection*{First in First out (FIFO)} % First Come First Serve (FCFS)

It has been assumed that arriving jobs should be served according to arrival time.

\begin{table}[H]
	\centering
	\caption{FIFO}
	\label{t2-fifo}
	\begin{tabular}{llllll}
		\hline
		\multicolumn{1}{|l|}{PID} & \multicolumn{1}{l|}{AT} & \multicolumn{1}{l|}{BT} & \multicolumn{1}{l|}{CT} & \multicolumn{1}{l|}{TAT} & \multicolumn{1}{l|}{WT} \\ \hline
		1                         & 0                       & 100                     & 100                     & 100                      & 0                       \\
		2                         & 1                       & 10                      & 110                     & 109                      & 99                      \\
		3                         & 2                       & 20                      & 130                     & 128                      & 108                     \\
		4                         & 3                       & 1                       & 131                     & 128                      & 127                     \\
		5                         & 10                      & 1                       & 131                     & 122                      & 121                     \\
		6                         & 20                      & 100                     & 232                     & 212                      & 112                    
	\end{tabular}
\end{table}

Thus the average turnaround time will be $100+109+128+128+122+212/6=~133.1667$

And the average response time also called waiting time will be $0+99+108+127+121+112/6=94.5$

\subsection*{Shortest Job First (SJF)}

\begin{table}[H]
	\centering
	\caption{SJF}
	\label{t2-sjf}
	\begin{tabular}{llllll}
		\hline
		\multicolumn{1}{|l|}{PID} & \multicolumn{1}{l|}{AT} & \multicolumn{1}{l|}{BT} & \multicolumn{1}{l|}{CT} & \multicolumn{1}{l|}{TAT} & \multicolumn{1}{l|}{WT} \\ \hline
		1                         & 0                       & 100                     & 100                     & 100                      & 0                       \\
		2                         & 1                       & 10                      & 112                     & 111                      & 101                     \\
		3                         & 2                       & 20                      & 132                     & 130                      & 110                     \\
		4                         & 3                       & 1                       & 101                     & 98                       & 97                      \\
		5                         & 10                      & 1                       & 102                     & 92                       & 91                      \\
		6                         & 20                      & 100                     & 232                     & 212                      & 112                    
	\end{tabular}
\end{table}


The average turnaround time will be $110+111+130+98+92+212/6=~123.83$

And the average response time will be

$0+101+110+97+91+112/6=~85.167$


\subsection*{Shortest Time to Completion First (STCF)}

\begin{table}[H]
	\centering
	\caption{STCF}
	\label{t2-stcf}
	\begin{tabular}{llllll}
		\hline
		\multicolumn{1}{|l|}{PID} & \multicolumn{1}{l|}{AT} & \multicolumn{1}{l|}{BT} & \multicolumn{1}{l|}{CT} & \multicolumn{1}{l|}{TAT} & \multicolumn{1}{l|}{WT} \\ \hline
		1                         & 0                       & 100                     & 132                     & 132                      & 32                      \\
		2                         & 1                       & 10                      & 13                      & 12                       & 2                       \\
		3                         & 2                       & 20                      & 33                      & 31                       & 11                      \\
		4                         & 3                       & 1                       & 4                       & 1                        & 0                       \\
		5                         & 10                      & 1                       & 11                      & 1                        & 0                       \\
		6                         & 20                      & 100                     & 232                     & 132                      & 32                     
	\end{tabular}
\end{table}

The average turnaround time will be $132+12+31+1+1+212/6=~64.834$

And the average response time will be $32+2+11+0+0+112/6=~26.167$

\subsection*{Round Robin (RR)}

timeslice is assumed to also mean timequantum with a time cutting off process each one second.

\begin{table}[H]
    \centering
    \caption{RR}
    \label{t2-rr}
    \begin{tabular}{llllll}
        \hline
        \multicolumn{1}{|l|}{PID} & \multicolumn{1}{l|}{AT} & \multicolumn{1}{l|}{BT} & \multicolumn{1}{l|}{CT} & \multicolumn{1}{l|}{TAT} & \multicolumn{1}{l|}{WT} \\ \hline
        1                         & 0                       & 100                     & 228                     & 228                      & 128                     \\
        2                         & 1                       & 10                      & 36                      & 35                       & 25                      \\
        3                         & 2                       & 20                      & 71                      & 69                       & 49                      \\
        4                         & 3                       & 1                       & 6                       & 3                        & 2                       \\
        5                         & 10                      & 1                       & 13                      & 3                        & 2                       \\
        6                         & 20                      & 100                     & 235                     & 215                      & 115                    
    \end{tabular}
\end{table}

The average turnaround time will be $228+35+69+3+3+215/6=~92.167$

And the average response time will be $128+25+49+2+2+115/6=~53.5$

If there is found mistakes in this particular algorithm pictures of run by hand can be found in appendix.



\section*{Theoretical 3: FAT-based file systems}

\subsection*{a) Allocate a file F1 of size 5,000 bytes. Show the FAT table}


\begin{figure}[H]
    \caption{FAT Table of F1 containing 5,000 bytes}
    \centering
    \includegraphics[scale=0.5]{figures/t3-a.png}
\end{figure}



\subsection*{b) Allocate a file F2 of size 20,000 bytes. Show the FAT table}

\begin{figure}[H]
    \caption{FAT Table of F2 containing 20,000 bytes}
    \centering
    \includegraphics[scale=0.5]{figures/t3-b.png}
\end{figure}



\subsection*{c) Extend F1 with an additional 5,000 bytes. Show the FAT table}

\begin{figure}[H]
    \caption{FAT Table of F1 containing 10,000 bytes}
    \centering
    \includegraphics[scale=0.5]{figures/t3-c.png}
\end{figure}



\subsection*{d) Shorten F2 to 10,000 bytes. Show the FAT table}

\begin{figure}[H]
    \caption{FAT Table of F2 shortened by 10,000 bytes}
    \centering
    \includegraphics[scale=0.5]{figures/t3-d.png}
\end{figure}



\subsection*{e) Allocate a file F3 of size 20,000 bytes. Show the FAT table}

\begin{figure}[H]
    \caption{FAT Table of file F3 by 20,000 bytes}
    \centering
    \includegraphics[scale=0.5]{figures/t3-e.png}
\end{figure}



\subsection*{f) How much space (in bytes) can the FAT manage in total?}
% Assuming one block for the FAT, how much space (in bytes) can we
Assuming one block of the FAT system to simply be as stated in the given text 4096, and that one block can only contain one pointer in the end.

The answer will be block $4096$ subtracted with pointer of $4$ bytes $= 4092$

\textit{I am convinced that my answer is incorrect, though I am unable to provide a better one.}


\subsection*{g) What are some problems with caching the FAT?}
\textit{Using FAT can result in a significant number of disk seeks, unless the FAT is cached. What are some problems with caching the FAT? Contrast this with caching the directory control block.}

Firstly as seen above, when a file is stored and later extended, the meantime allows for another file to be placed in the following block, and thus forcing rest of the file to be stored at a completely different part of the disk and thereby making load time of a normal disk-based hard drive very inefficient.



\section*{Practical 1: Monitors in KUDOS}




\section*{Practical 2: Bunny Allocation}



\section*{Appendix}




%\newpage
%\bibliography{mybib}
%\bibliographystyle{ieeetr}
\end{document}
